\mychapter{1}{Lesson 1} %180926
Talking cryptography is usually done in the `confidentiality' realm, where two characteristics in a communication channel are desirable: it must be \emph{secret}, and \emph{authentic}.



\section{Secret communication}
Modern confidentiality/authentication systems are designed following \emph{Kerckhoffs's principle}, which states that a secure system shall only rely on the encryption keys and not on the algorithm's secrecy.
The principle is often condensed as \emph{`no security by obscurity'}.

In symmetric-key encryption (\ske), the same cryptographic key is used for both encryption of plaintext and decryption of ciphertext.
However, sharing the key between two parties without the risk of eavesdropping is a costly operation.

\begin{figure}[h]
	\centering
	\begin{tikzpicture}
	\begin{pgfonlayer}{nodelayer}
		\node [style=none] (11) at (-1, 0.5) {};
		\node [style=none] (12) at (1, 0.5) {};
		\node [style=none] (20) at (0, -1.35) {$k \in \K$};
		\node [style=none] (21) at (0, -1.1) {};
		\node [style=none] (22) at (0, 0.75) {$c \in \C$};
		\node [style=mediumbox] (23) at (-2, 0.5) {$\Enc$};
		\node [style=mediumbox] (24) at (2, 0.5) {$\Dec$};
		\node [style=none] (25) at (-3.75, 0.5) {};
		\node [style=none] (26) at (3.75, 0.5) {};
		\node [style=none] (27) at (-3, 0.5) {};
		\node [style=none] (28) at (3, 0.5) {};
		\node [style=none] (29) at (-4.5, 0.5) {$m \in \M$};
		\node [style=none] (30) at (4, 0.5) {$m$};
		\node [style=none] (35) at (2, -0.6) {};
		\node [style=none] (36) at (2, -0.1) {};
		\node [style=none] (37) at (-2, -0.6) {};
		\node [style=none] (38) at (0, -0.6) {};
		\node [style=none] (39) at (-2, -0.1) {};
	\end{pgfonlayer}
	\begin{pgfonlayer}{edgelayer}
		\draw [style=leftarrow] (11.center) to (12.center);
		\draw [style=leftarrow] (25.center) to (27.center);
		\draw [style=leftarrow] (28.center) to (26.center);
		\draw (21.center) to (38.center);
		\draw [style=leftarrow] (38.center)
			 to (37.center)
			 to (39.center);
		\draw [style=leftarrow] (38.center)
			 to (35.center)
			 to (36.center);
	\end{pgfonlayer}
\end{tikzpicture}

	\caption{Symmetric-key encryption}
\end{figure}


\subsection{Syntax}
The typical objects defined and used throughout cryptography discourses are:
%
\begin{itemize}[parsep=0pt]
    \item the key space $\K$
    \item the message space $\M$
    \item the ciphertext space $\C$
    \item the encryption function $\begin{aligned}[t]\Enc \, \tc \, & \K \times \M \to \C \\ & (k, m) \mapsto c\end{aligned}$
    \item the decryption function $\begin{aligned}[t]\Dec \, \tc \, & \K \times \C \to \M \\ & (k, c) \mapsto m \, . \end{aligned}$
\end{itemize}
\Enc{} and \Dec{} form a cryptographic scheme, which must be correct:
\[
    \forall m \in \M \ \forall k \in \K \ \Enc(k, m) = c \iff \Dec(k, c) = m \, .
\]


\subsection{Perfect secrecy}
\begin{definition}[Shannon's `Perfect secrecy']\label{def:perfect-secrecy}
	Let $M$ be any distribution over the message space $\M$, $C$ be any distribution over the cyphertext space $\C$ and $K$ be a uniform distribution over the key space $\K$.
    Then, the cryptosystem (\Enc, \Dec) is deemed \emph{perfectly secret} iff $\forall M, m \in \M \ \forall c \in C$
%
    \begin{equation}\label{eq:perfect-secrecy-1}
    	 \Pr[M \evaluatesto m] = \Pr[M \evaluatesto m| C \evaluatesto c] \, .
    \end{equation}
\end{definition}

The cyphertext reveals nothing about the message, thus making the encryption method uncrackable even by adversaries with infinite computational power.
This definition does not involve the encryption key because it must hold for all keys in the key space.
It can be rephrased in different ways, bringing more details to light:
%
\begin{enumerate}
    \item equation \ref{eq:perfect-secrecy-1}
%
    \item $M \bot C$ ($M$ and $C$ are independent)\hfill\refstepcounter{equation}\textup{(\theequation)}\label{eq:perfect-secrecy-2}
%
    \item $k \pickUAR\footnote{\href{https://math.stackexchange.com/a/1576505}{What does this dollar sign over arrow in function mapping mean?}} \K \implies \Pr[\Enc(k, m_1) \evaluatesto c] = \Pr[\Enc(k, m_2) \evaluatesto c]$.\hfill\refstepcounter{equation}\textup{(\theequation)}\label{eq:perfect-secrecy-3}
\end{enumerate}
    
\begin{proposition}
    All three previous statements are equivalent.
\end{proposition}
\begin{proof}
    The proof is structured as a cyclic implication between the three definitions:
%
    \begin{itemize}
    	\item $(1) \implies (2)$
        \begin{flalign*}
            \Pr[C \evaluatesto c \wedge M \evaluatesto m] &= \Pr[C \evaluatesto c] \Pr[M \evaluatesto m|C \evaluatesto c] & \text{(Cond. prob.)} \\
            &= \Pr[C \evaluatesto c] \Pr[M \evaluatesto m] & \text{(Using \ref{eq:perfect-secrecy-1})}
        \end{flalign*}
    	Which implies that $M$ and $C$ are independent.
%
        \item $(2) \implies (3)$ fix $M$. Then,
        \begin{flalign*}% TODO: alignment
            \Pr[\Enc(K, m_1) \evaluatesto c] &= \Pr[\Enc(K, M) \evaluatesto c| M \evaluatesto m_1] & \\
            &= \Pr[C \evaluatesto c|M \evaluatesto m_1] & \text{($C$ def.)} \\
            &= \Pr[C \evaluatesto c] & \text{(Using \ref{eq:perfect-secrecy-2})} \\
            &= \Pr[C \evaluatesto c|M \evaluatesto m_2] & \text{(Reverse steps with $m_2$ instead of $m_1$)} \\
            &= \Pr[\Enc(K, M) \evaluatesto c| M \evaluatesto m_2] & \\
            &= \Pr[\Enc(K, m_2) \evaluatesto c] &
        \end{flalign*}
%
        \item $(3) \implies (1)$
        \begin{flalign*}
            \Pr[C \evaluatesto c] &= \sum_{m}\Pr[C \evaluatesto c \wedge M \evaluatesto m] & \text{(Total prob.)} \\
            &= \sum_{m}\Pr[\Enc(K, M) \evaluatesto c \wedge M \evaluatesto m] & \text{($C$ def.)} \\
            &= \sum_{m}\Pr[\Enc(K, M) \evaluatesto c|M \evaluatesto m]\Pr[M \evaluatesto m] & \text{(Cond. prob.)} \\
            &= \sum_{m}\Pr[\Enc(K, m) \evaluatesto c]\Pr[M \evaluatesto m] & \text{(Prob. collapse)} \\
            &= \Pr[\Enc(K, m') \evaluatesto c] \sum_{m}\Pr[M \evaluatesto m] & \text{(Using \ref{eq:perfect-secrecy-3})} \\
            &= \Pr[\Enc(K, m') \evaluatesto c] & \text{(Total prob.)} \\
            &= \Pr[\Enc(K, M) \evaluatesto c|M \evaluatesto m'] & \text{(Cond. prob.)} \\
            &= \Pr[C \evaluatesto c|M \evaluatesto m'] & \text{($C$ def.)}
        \end{flalign*}
%
        Then,
        \begin{flalign*}
            & \Pr[C \evaluatesto c] = \Pr[C \evaluatesto c|M \evaluatesto m] & \\
            &\implies \Pr[C \evaluatesto c] = \Pr[M \evaluatesto m|C \evaluatesto c]\frac{\Pr[C \evaluatesto c]}{\Pr[M \evaluatesto m]} & \text{(Bayes's th.)} \\
            &\implies \Pr[M \evaluatesto m] = \Pr[M \evaluatesto m|C \evaluatesto c] &
        \end{flalign*}
    \end{itemize}
%
    All three definitions for perfect secrecy are equivalent.
\end{proof}


\subsection{One-time pad (\otp)}
The one-time pad (\otp) is a cryptographic scheme such that
%
\begin{itemize}
	\item $\K = \M = \C = \Bool^l$
	\item $K \sim \unif{\K}$
    \item $\Enc(k, m) = k \xor m$
    \item $\Dec(k, c) = k \xor c$.
\end{itemize}
%
The scheme is correct because $\Dec(k, \Enc(k, m)) = \Dec(k, k \xor m) = k \xor k \xor m = m$.

\begin{theorem}\label{th:otp}
    \otp{} is perfectly secret.
\end{theorem}
\begin{proof}
    We have that $\forall m_1, m_2 \in \M$ and $\forall c \in \C$
    \begin{flalign*}% TODO: alignment
        \Pr[\Enc(K, m_1) \evaluatesto c] &= \Pr[K \xor m_1 \evaluatesto c] & \\
        &= \Pr[K \evaluatesto c \xor m_1] & \\
        &= |\K|^{-1} & \text{($K$ is uniform)} \\
        &= \dots & \text{(Reverse steps with $m_1$ instead of $m_2$)} \\
        &= \Pr[\Enc(K, m_2) \evaluatesto c] &  
    \end{flalign*}
    This satisfies equation \ref{eq:perfect-secrecy-3}.
\end{proof}

%TODO OLD NOTE: Observation: k is truly random, but fixed, compare with Pi_\xor's encryption routine: from a security standpoint, nothing changes! actualy, \Pi_\xor may be weaker bruteforce wise than OTP!

By observing the proof of theorem \ref{th:otp}, we can  gain a deeper insight of \otp:
%
\begin{enumerate}
    \item $|k| = |m|$ (key and message sizes match)
    \item if a key is used multiple two or more times, an attacker may exploit \textsc{xor}'s idempotency to extract valuable information from the ciphertexts\footnote{A cryptographic algorithm is said to be \emph{malleable} if there exists a function $f$ that transforms a ciphertext into another ciphertext which decrypts to a related plaintext (i.e. $f(c) = \Enc(k, f(m))$). This concept will be explored further.}:
    \[
        c_1 = k \xor m_1 \wedge c_2 = k \xor m_2 \implies c_1 \xor c_2 = m_1 \xor m_2
    \]
\end{enumerate}
%
Combined with the fact that keys must be pre-emptively shared in a secure fashion, these problems make \otp{} impractical.


\subsection{Size of the key space}
It is possible to generalise the analysis on \otp{} to all perfectly secret schemes.

\begin{theorem}
    In any perfectly secret \ske, the size of the key space must be greater than or equal to the size of the message space:
    \[
    	|\K| \geq |\M| \, .
    \]
\end{theorem}
\begin{proof}
    Perfect secrecy will be disproved by breaking the first definition. \\
    Define $M \sim \unif{\M}$ and $c \in \C : \Pr[C \evaluatesto c] > 0$.
    Let $\D = \{\Dec(k, c) : k \in \K\}$ be the set of all images of the decryption routine with all keys in $\K$ and assume $|\K| < |\M|$ (negation of the thesis).
    Then, using the definition of $\D$ we have
    \[
        |D| \leq |\K| < |\M| \implies \exists m \in \M \setminus \D \, .
    \]
    
    \begin{figure}[h]
    	\centering
    	\begin{tikzpicture}
	\begin{pgfonlayer}{nodelayer}
		\node [style=nodecircle, minimum size=1.5cm] (1) at (0, 0) {};
		\node [style=nodecircle, minimum size=3cm] (0) at (0, 0) {$\D$};
		\node [style=none] (2) at (1.1, 0) {$\M$};
		\node [style=point, label={left:$m$}] (3) at (0, -1.15) {};
	\end{pgfonlayer}
\end{tikzpicture}

    	\caption{Visual representation of $\D$, $\M$ and $m$}
    \end{figure}

    Fix $m$. Using $M$'s definition, we have $\Pr[M \evaluatesto m] = |\M|^{-1}$.
    Since $m \notin \D$, there can be no key in $\K$ such that $\Dec(k, c) = m$.
    We can prove $\Pr[M \evaluatesto m|C \evaluatesto c] = 0$ by observing that $C$ strictly distributes over $\D$, in which $m$ is not present.
    
    In conclusion, we have
    \[
        \Pr[M \evaluatesto m|C \evaluatesto c] = 0 \neq |\M|^{-1} = \Pr[M \evaluatesto m]
    \]
    that clearly violates the first definition of perfect secrecy.
\end{proof}

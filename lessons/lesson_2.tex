\mychapter{2}{Lesson 2} %180928

\section{Message authentication codes (\Mac)}
A message authentication code (\Mac) allows to verify the integrity and the source of a message.

\begin{figure}[h]
	\centering
	\begin{tikzpicture}
	\begin{pgfonlayer}{nodelayer}
		\node [style=none] (11) at (-0.25, -0.75) {};
		\node [style=none] (12) at (2.25, 0.5) {};
		\node [style=none] (20) at (3.25, -2.6) {$k \in \K$};
		\node [style=none] (21) at (3.25, -2.35) {};
		\node [style=none] (22) at (1.5, 0.75) {$(m, \tau)$};
		\node [style=mediumbox] (23) at (-1.25, -0.75) {$\Tag$};
		\node [style=mediumbox] (24) at (3.25, 0.5) {$\Vrfy$};
		\node [style=none] (25) at (-3.75, 0.5) {};
		\node [style=none] (26) at (5, 0.5) {};
		\node [style=none] (27) at (-2.25, -0.75) {};
		\node [style=none] (28) at (4.25, 0.5) {};
		\node [style=none] (29) at (-4.5, 0.5) {$m \in \M$};
		\node [style=none] (30) at (5.75, 0.5) {$\BoolSym$};
		\node [style=none] (36) at (3.25, -0.1) {};
		\node [style=none] (40) at (-3, 0.5) {};
		\node [style=none] (41) at (-3, -0.75) {};
		\node [style=none] (42) at (-1.25, -1.35) {};
		\node [style=none] (44) at (3.25, -1.85) {};
		\node [style=none] (45) at (-1.25, -1.85) {};
		\node [style=none] (46) at (0.5, -0.75) {};
		\node [style=nodeoperator] (47) at (0.5, 0.5) {+};
		\node [style=none] (48) at (0.75, -1) {$\tau \in \Tau$};
	\end{pgfonlayer}
	\begin{pgfonlayer}{edgelayer}
		\draw [style=leftarrow] (28.center) to (26.center);
		\draw [style=leftarrow] (40.center)
			 to (41.center)
			 to (27.center);
		\draw [style=leftarrow] (21.center) to (36.center);
		\draw [style=leftarrow] (44.center)
			 to (45.center)
			 to (42.center);
		\draw [style=leftarrow] (11.center)
			 to (46.center)
			 to (47);
		\draw [style=leftarrow] (40.center) to (47);
		\draw [style=leftarrow] (47) to (12.center);
		\draw (25.center) to (40.center);
	\end{pgfonlayer}
\end{tikzpicture}

	\caption{Message authentication code}
\end{figure}


\subsection{Syntax}
The typical objects defined and used throughout message-authentication discourses are:
%
\begin{itemize}[parsep=0pt]
	\item the key space $\K$
%
	\item the message space $\M$
%
	\item the tag space $\Tau$
%
	\item the random variable $K \sim \unif{\K}$
%
    \item the tag function $\begin{aligned}[t]\Tag \, \tc \, & \K \times \M \to \Tau \\ & (k, m) \mapsto \tau\end{aligned}$
    \item the verify function $\begin{aligned}[t]\Vrfy \, \tc \, & \K \times \M \times \Tau \to \BoolSym \\ & (k, m, \tau) \mapsto \Tag(k, m) = \tau \, .\end{aligned}$
\end{itemize}
%
\Tag{} and \Vrfy{} form a message-authentication scheme, which must be correct:
\[
	\forall m \in \M \ \forall k \in \K \ \forall \tau \in \Tau \ \Vrfy(k, m, \tau) = \top \iff \Tag(k, m) = \tau \, .
\]


\subsection{Properties}
Since \Tag{} is deterministic, there exists a canonical \Vrfy{} that always produce the correct output.
A \Mac{} is unforgeable if it is computationally infeasible to find a valid tag of the given message without knowledge of the key.
However, unforgeability does not imply that an adversary cannot forge tags: in the worst case, we assume the adversary can forge the tag of any message except the given one.

\begin{definition}[$\eps$-statistical one-time security]
	We say that a \Mac{} has $\eps$-statistical one-time security if $\forall m, m' \in \M : m \neq m' \ \forall \tau, \tau' \in \Tau$
	\begin{equation}
		\Pr[\Tag(K,m') \evaluatesto \tau' | \Tag(K,m) \evaluatesto \tau] \leq \eps \, .
	\end{equation}
\end{definition}

Suppose a computationally unbounded adversary chooses $m \in \M$ and obtains $\tau = \Tag(k, m)$.
His goal is to forge a tag on a new message $m' \neq m$, i.e. finding a tag $\tau'$ such that $\Tag(k,m') = \tau'$, with no knowledge of $k$.


\subsection{Pairwise independent hashing}
\begin{definition}[Pairwise independence]
    A family of (hash) functions $H \triangleq \footnote{\href{https://math.stackexchange.com/questions/522864/what-is-the-symbol-triangleq}{What is the symbol $\triangleq$?}} \{h_s:\M \to \Tau\}_{s \in \mathcal{S}}$ is said to be pairwise independent iff, for any two distinct messages $m$ and $m'$, we have that $(h_s(m), h_s(m'))$ is uniform over $\Tau^2$ for a random choice of seed $s \in \mathcal{S}$:
    \[
        \forall m, m' : m \neq m' \ \forall \tau, \tau' \in \Tau \ \Pr[h_s(m) \evaluatesto \tau \wedge h_s(m') \evaluatesto \tau'] = \frac{1}{|\Tau|^2} \, .
    \]
\end{definition}


\begin{theorem}
	Let $p$ be a prime number and let $\M = \Tau = \Z_p = \{0, 1, \dots, p-1\}$.
	Then, we can define the group $(\Z_p, +)$.
	Consider the function $h_{a,b}(x) = ax + b \mod p$ where $a, b \in \Z_p$.
    The family $H = \{h_{a,b}\}$ is pairwise independent.
\end{theorem}
\begin{proof}
	Fix $m, m' \in \Z_p : m \neq m'$ and $\tau, \tau' \in \Z_p$.
	Assume that $a$ and $b$ are random in $\Z_p$. Then,
    \begin{flalign*}
        \Pr_{a,b}[h_{a,b}(m) \evaluatesto \tau \wedge h_{a,b}(m') \evaluatesto \tau'] &= \Pr_{a,b}[am+b \evaluatesto \tau \wedge am'+b \evaluatesto \tau'] \\
        &= \Pr_{a,b}\left[\begin{pmatrix}
        	m & 1 \\
            m' & 1
        \end{pmatrix} \cdot \begin{pmatrix}
            a \\
            b
        \end{pmatrix} \evaluatesto \begin{pmatrix}
            \tau \\
            \tau'
        \end{pmatrix}\right] \\
        &= \Pr_{a,b}\left[\begin{pmatrix}
            a \\
            b
        \end{pmatrix} \evaluatesto \begin{pmatrix}
            m & 1 \\
            m' & 1
        \end{pmatrix}^{-1} \cdot \begin{pmatrix}
            \tau \\
            \tau'
        \end{pmatrix}\right] \\
        &= \frac{1}{|\Z_p|^2} \\
        &= \frac{1}{p^2}
    \end{flalign*}
    where $\Pr_{a,b}$ denotes the probability over $a$ and $b$ random in $\Z_p$.
\end{proof}

\begin{theorem}
    Let $\Tag(K,m) = h_k(m)$ for $\H = \{h_k : \M \to \Tau \}_k$ be pairwise independent. 
    Then, the \Mac{} is $\frac{1}{|\Tau|}$-statistical one-time secure. 
\end{theorem}

\begin{proof}
	\begin{equation*}
		\begin{aligned}
			&\forall m \in \M \ \Pr[\Tag(K, m) \evaluatesto \tau] = \Pr_k[h_k(m) \evaluatesto \tau] = \frac{1}{|\Tau|} \\
			&\forall m, m' \in \M : m \neq m' \ \forall \tau \in \Tau \ \Pr[\Tag(K, m) \evaluatesto \tau \wedge \Tag(K, m') \evaluatesto \tau'] = \frac{1}{|\Tau|^2} \\
			&\begin{aligned}
				\Pr[\Tag(K, m') \evaluatesto \tau' | \Tag(K, m) \evaluatesto \tau] &= \frac{\Pr[Mac(K,m)=\tau \wedge Mac(K,m')=\tau']}{Pr[Mac(K,m)=\tau]} \\
				&= \frac{|\Tau|^{-2}}{|\Tau|^{-1}} \\
				&= \frac{1}{|\Tau|}
			\end{aligned}
		\end{aligned}
	\end{equation*}
\end{proof}

\begin{lemma}
    A $2^{-\lambda}$-statistical $t$-times secure \Mac{} has a key of size $(t+1) \lambda$ for all $\lambda \in \Naturals$.
\end{lemma}